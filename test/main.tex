\documentclass[conference]{IEEEtran}
\IEEEoverridecommandlockouts
% The preceding line is only needed to identify funding in the first footnote. If that is unneeded, please comment it out.
\usepackage{cite}
\usepackage{amsmath,amssymb,amsfonts}
\usepackage{algorithmic}
\usepackage{graphicx}
\usepackage{textcomp}
\usepackage{xcolor}
% \def\BibTeX{{\rm B\kern-.05em{\sc i\kern-.025em b}\kern-.08em
%     T\kern-.1667em\lower.7ex\hbox{E}\kern-.125emX}}
\begin{document}

\title{Melecular retrosynthesis based on computer\\
\thanks{Identify applicable funding agency here. If none, delete this.}
}

\author{\IEEEauthorblockN{1\textsuperscript{st} Zheyi Cai}
\IEEEauthorblockA{\textit{Department of Computer Science} \\
\textit{Xiamen University}\\
Xiamen, China \\
23020231154166@stu.xmu.edu.cn}
}



\maketitle 

\begin{abstract}
Molecular retrosynthesis, a pivotal technique in synthetic chemistry, involves devising pathways to synthesize target molecules from simpler precursors\cite{label1}. Historically, this task heavily relied on the expertise and intuition of chemists. However, the last five decades have seen a surge in the application of computational tools to address retrosynthetic challenges. Particularly in recent years, advancements in computing capabilities, data availability, and data-driven algorithms have further propelled the adoption of computer-aided retrosynthesis. This paper delves into the evolution and current state of computer-aided retrosynthetic design, emphasizing the integration of machine learning techniques\cite{label2} and the distinction between template-based and template-free methodologies\cite{label3}. We also explore its applications in drug design, novel synthetic route discovery, and the design of biologically active compounds. The potential challenges and future prospects of automated synthetic planning are also discussed.
\end{abstract}

\begin{IEEEkeywords}
retrosynthesis, template, machine learning, tools, outlook
\end{IEEEkeywords}

\section{Introduction}
1 
\vspace{12pt}
\color{black}
\bibliographystyle{IEEEtran}
\bibliography{references}
\end{document}